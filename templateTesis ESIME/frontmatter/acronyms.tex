\usepackage[intoc, spanish]{nomencl}
\makenomenclature

%% grupos de nomenclatura
% -----------------------------------------
\usepackage{etoolbox}
\renewcommand\nomgroup[1]{%
  \item[\bfseries
  \ifstrequal{#1}{A}{Constantes físicas}{%
  \ifstrequal{#1}{B}{Conjuntos}{%
  \ifstrequal{#1}{C}{Símbolos}{}}}%
]}
%% nomenclatura con unidades
%----------------------------------------------
\newcommand{\nomunit}[1]{%
\renewcommand{\nomentryend}{\hspace*{\fill}#1}}

\nomenclature[C]{$\vec{E}$}{{Campo eléctrico}\nomunit{\unit{V/m}}}
\nomenclature[C]{$P_{in}$}{{Potencia de entrada}\nomunit{\unit{\W}}}
\nomenclature[C]{$P_A$}{{Potencia del actuador}\nomunit{\unit{\W}}}
\nomenclature[C]{$P_{FM}$}{{Potencia fluido mecánica}\nomunit{\unit{\W}}}

\DeclareAcronym{rans}{
  short = RANS ,
  long  = ecuaciones promediadas de Navier-Stokes,
  foreign = Reynolds-averaged Navier–Stokes equations,
  tag = acro
}

\DeclareAcronym{dns}{
  short = DNS ,
  long  = simulación numérica directa,
  foreign = Direct Numerial Simulation,
  tag = acro
}

\DeclareAcronym{les}{
  short = LES ,
  long  = simulación de grandes escalas,
  foreign = Large Eddy Simulation,
  tag = acro
}